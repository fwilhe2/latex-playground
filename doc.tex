\documentclass[a4paper]{article}
\usepackage{ngerman}
\begin{document}
\author{K. Einstein}
\title{Formeln in Fachartikeln}
\maketitle
\section{Einleitung}
Einer Theorie zufolge enthalten 
wissenschaftliche Artikel Formeln.
\section{Methode}
Um eine 2 in einer \LaTeX-Formel 
hochzustellen, stellt man der 
Ziffer ein \verb+^+ voran.
\section{Ergebnis}
So erhalten wir \(E=mc^2\).
\section{Diskussion}
Zum selben Ergebnis kam 
\textsc{A. Einstein}.
\section{Fazit}
Wir konnten die Theorie best"atigen.
\section*{Literatur}
E. Robert Schulman: ``How to Write a 
Scientific Paper.'' \textit{Annals of 
Improbable Research}, Vol.~2, No.~5
(1996).
\end{document}
